\documentclass{article}\usepackage[]{graphicx}\usepackage[]{color}
%% maxwidth is the original width if it is less than linewidth
%% otherwise use linewidth (to make sure the graphics do not exceed the margin)
\makeatletter
\def\maxwidth{ %
  \ifdim\Gin@nat@width>\linewidth
    \linewidth
  \else
    \Gin@nat@width
  \fi
}
\makeatother

\definecolor{fgcolor}{rgb}{0.345, 0.345, 0.345}
\newcommand{\hlnum}[1]{\textcolor[rgb]{0.686,0.059,0.569}{#1}}%
\newcommand{\hlstr}[1]{\textcolor[rgb]{0.192,0.494,0.8}{#1}}%
\newcommand{\hlcom}[1]{\textcolor[rgb]{0.678,0.584,0.686}{\textit{#1}}}%
\newcommand{\hlopt}[1]{\textcolor[rgb]{0,0,0}{#1}}%
\newcommand{\hlstd}[1]{\textcolor[rgb]{0.345,0.345,0.345}{#1}}%
\newcommand{\hlkwa}[1]{\textcolor[rgb]{0.161,0.373,0.58}{\textbf{#1}}}%
\newcommand{\hlkwb}[1]{\textcolor[rgb]{0.69,0.353,0.396}{#1}}%
\newcommand{\hlkwc}[1]{\textcolor[rgb]{0.333,0.667,0.333}{#1}}%
\newcommand{\hlkwd}[1]{\textcolor[rgb]{0.737,0.353,0.396}{\textbf{#1}}}%

\usepackage{framed}
\makeatletter
\newenvironment{kframe}{%
 \def\at@end@of@kframe{}%
 \ifinner\ifhmode%
  \def\at@end@of@kframe{\end{minipage}}%
  \begin{minipage}{\columnwidth}%
 \fi\fi%
 \def\FrameCommand##1{\hskip\@totalleftmargin \hskip-\fboxsep
 \colorbox{shadecolor}{##1}\hskip-\fboxsep
     % There is no \\@totalrightmargin, so:
     \hskip-\linewidth \hskip-\@totalleftmargin \hskip\columnwidth}%
 \MakeFramed {\advance\hsize-\width
   \@totalleftmargin\z@ \linewidth\hsize
   \@setminipage}}%
 {\par\unskip\endMakeFramed%
 \at@end@of@kframe}
\makeatother

\definecolor{shadecolor}{rgb}{.97, .97, .97}
\definecolor{messagecolor}{rgb}{0, 0, 0}
\definecolor{warningcolor}{rgb}{1, 0, 1}
\definecolor{errorcolor}{rgb}{1, 0, 0}
\newenvironment{knitrout}{}{} % an empty environment to be redefined in TeX

\usepackage{alltt}
\usepackage[sc]{mathpazo}
\usepackage[T1]{fontenc}
\usepackage{amsmath}
\usepackage[utf8]{inputenc}
\usepackage{geometry}
\geometry{verbose,tmargin=1.5cm,bmargin=2.5cm,lmargin=2.5cm,rmargin=2.5cm}
\setcounter{secnumdepth}{2}
\setcounter{tocdepth}{2}
\usepackage{url}
\usepackage[unicode=true,pdfusetitle,
 bookmarks=true,bookmarksnumbered=true,bookmarksopen=true,bookmarksopenlevel=2,
 breaklinks=false,pdfborder={0 0 1},backref=false,colorlinks=false]
 {hyperref}
\hypersetup{
 pdfstartview={XYZ null null 1}}
\IfFileExists{upquote.sty}{\usepackage{upquote}}{}
\begin{document}



\title{Laboratorio de R}


\author{Curso: Introducci\'on a la Estad\'istica y Probabilidades CM-274}
\date{}
\maketitle

\vspace{0.3cm}


\textbf{Lecturas Importantes }
\begin{enumerate}
\item  \url{ http://thecodelesscode.com/contents}, un compendio de f\'abulas ilustradas que hablan del trabajo y arte de programar. La mayor parte de historias se encuentra tanto en ingl\'es como en espa\~nol, franc\'es, italiano y alem\'an.
\item \textit{97 cosas que todo programador deber\'ia saber}, una colecci\'on de casi  un centenar de art\i'culos especializados repletos de consejos y datos intersantes \url{http://programmer.97things.oreilly.com/wiki/index.php/Contributions_Appearing_in_the_Book}. Es demasiado \'util.
\end{enumerate}
%{\normalsize Los c\'odigos, se presentaran impresos,  o como un archivo con extensi\'on $.R$, mostrando ejemplos de su ejecuci\'on.}
\setlength{\unitlength}{1in}

\begin{picture}(6,.1) 
\put(0,0) {\line(1,0){6.25}}         
\end{picture}

\vspace{0.2cm}

{\Large Preguntas }


\vspace{0.3cm}

\begin{enumerate}
\item
\begin{itemize}
\item Cu\'al es el valor producido por la siguiente expresi\'on:
\begin{knitrout}
\definecolor{shadecolor}{rgb}{0.969, 0.969, 0.969}\color{fgcolor}\begin{kframe}
\begin{alltt}
\hlstd{> }\hlnum{1}\hlopt{:}\hlnum{6} \hlopt{*} \hlnum{1}\hlopt{:}\hlnum{2}
\end{alltt}
\end{kframe}
\end{knitrout}
Explica en detalle, como los valores son calculados
\item La funci\'on 
\begin{knitrout}
\definecolor{shadecolor}{rgb}{0.969, 0.969, 0.969}\color{fgcolor}\begin{kframe}
\begin{alltt}
\hlstd{> }\hlstd{f} \hlkwb{<-}\hlkwa{function}\hlstd{(}\hlkwc{x}\hlstd{,}\hlkwc{y}\hlstd{)\{}
\hlstd{+ }  \hlkwa{if}\hlstd{(y} \hlopt{>} \hlnum{0}\hlstd{)}
\hlstd{+ }    \hlstd{y} \hlopt{*}\hlkwd{sin}\hlstd{(x)}
\hlstd{+ }  \hlkwa{else}
\hlstd{+ }    \hlstd{x}\hlopt{*}\hlkwd{sin}\hlstd{(y)}
\hlstd{+ }\hlstd{\}}
\end{alltt}
\end{kframe}
\end{knitrout}
no soporta el \textbf{recycling}. Explica como puedes modificar la funci\'on para que si pueda soportarlo.
\end{itemize}

\item Escribe operaciones en R, para generar cada uno de los siguientes vectores
\begin{itemize}
\item El vector conteniendo los valores $ 1, -2, 3, -4, \dots, 99, -100$.
\item El vector conteniendo los primeros 100 valores del factorial.
\item El vector conteniendo las primeras $100$ potencias de $2$.
\end{itemize}
\item Supongamos que tenemos un conjunto de valores  num\'ericos $x$ de un vector $X$ y un conjunto diferente de valores  num\'ericos $y$ en un vector $Y$.

\begin{itemize}
\item Describe c\'omo calcular la distancia m\'inima entre un valor  $x$ y un valor $y$: $\min_{i,j}|x_i -y_j|$.
\item ?` C\'omo determinar el par de  \'indices $(i, j)$ para los que la distancia m\'inima definida anteriormente se alcanza?.
\end{itemize}
\item El vector \textbf{mes.long} es definido como 

\begin{knitrout}
\definecolor{shadecolor}{rgb}{0.969, 0.969, 0.969}\color{fgcolor}\begin{kframe}
\begin{alltt}
\hlstd{> }\hlstd{month.len} \hlkwb{=}
\hlstd{+ }\hlkwd{c}\hlstd{(}\hlnum{31}\hlstd{,} \hlnum{28}\hlstd{,} \hlnum{31}\hlstd{,} \hlnum{30}\hlstd{,} \hlnum{31}\hlstd{,} \hlnum{30}\hlstd{,}
\hlstd{+ }\hlnum{31}\hlstd{,} \hlnum{31}\hlstd{,} \hlnum{30}\hlstd{,} \hlnum{31}\hlstd{,} \hlnum{30}\hlstd{,} \hlnum{31}\hlstd{)}
\end{alltt}
\end{kframe}
\end{knitrout}
\begin{itemize}
\item 
Muestra c\'omo este vector se puede utilizar para generar todas las fechas del a\~no como un vector de $365$  cadenas en el formato \texttt{d/m/a}, donde \texttt{d, m} e \texttt{y} son n\'umeros.
\item Muestra c\'omo utilizar el vector de cadenas de la pregunta anterior para escribir una funci\'on que determina la fecha de  cualquier d\'ia del a\~no en el rango $[1, 365]$ (la funci\'on debe ser vectorizada).
\item Si el  primer d\'ia del $2007$ fue un lunes. Escribe una funci\'on que determina el d\'ia de la
semana para cualquier fecha en el a\~no $2007$.
\end{itemize}
\item 
\begin{itemize}
\item Encuentra esxpresiones en R para encontrar el epsilon de la m\'aquina. \url{https://en.wikipedia.org/wiki/Machine_epsilon}.
\item Reproduce el siguiente c\'odigo fuente en R, para mostrar  la siguiente  tabla de probabilidad de la \texttt{distribuci\'on est\'andar normal}. Explica el uso de la funci\'on \texttt{outer()}.
\begin{knitrout}
\definecolor{shadecolor}{rgb}{0.969, 0.969, 0.969}\color{fgcolor}\begin{kframe}
\begin{alltt}
\hlstd{> }\hlstd{id} \hlkwb{<-} \hlnum{0}\hlopt{:}\hlnum{4}
\hlstd{> }\hlstd{dn} \hlkwb{<-} \hlkwd{seq}\hlstd{(}\hlnum{0}\hlstd{,} \hlnum{.8}\hlstd{,} \hlkwc{by} \hlstd{=}\hlnum{.2}\hlstd{)}
\hlstd{> }\hlstd{p} \hlkwb{=} \hlkwd{outer}\hlstd{(id, dn,} \hlkwa{function}\hlstd{(}\hlkwc{x}\hlstd{,}\hlkwc{y}\hlstd{)} \hlkwd{pnorm}\hlstd{(x} \hlopt{+} \hlstd{y))}
\hlstd{> }\hlkwd{dimnames}\hlstd{(p)} \hlkwb{=} \hlkwd{list}\hlstd{(}\hlkwc{z} \hlstd{= id,} \hlstr{"Primer lugar decimal de z "} \hlstd{= dn)}
\hlstd{> }\hlstd{p} \hlkwb{=} \hlkwd{round}\hlstd{(p,} \hlnum{5}\hlstd{)}
\end{alltt}
\end{kframe}
\end{knitrout}
\end{itemize}
\item
\begin{itemize}
\item Dada una matriz num\'erica $X$, determinar el \'indice de la primera fila cuyos elementos son todos  n\'umeros positivos (y que no contienen  valores  NA). Resuelve usando la funci\'on \texttt{apply} y usando un bucle \texttt{for}.
\item Escribe una funci\'on llamada \texttt{nesimo.na (x,n)} que toma un vector $x$ y retorna 
\begin{itemize}
\item el \'indice de la \texttt{\'enesima} valor NA que ocurre en $x$ o
\item NA si hay menos de $n$ valores NA en el vector $x$. 
\end{itemize}
\end{itemize}
\item 
\begin{itemize}
\item Escribe una sencilla expresi\'on de R que devuelve un vector que contiene el elemento m\'as \mbox{peque\~no} de cada fila de una matriz $x$.
\item Muestra tres maneras diferentes de calcular las sumas de las filas de una matriz $x$. (La suma de la i-\'esimo fila es la suma de los elementos de la fila i-\'esima.)
\end{itemize}
\item La funci\'on exponencial es definida por la serie de potencia

\[
\exp x = 1 + x + \frac{x^2}{2!} + \frac{x^3}{3!} + \frac{x^4}{4!} + \cdots
\]
Escribe una funci\'on en R vectorizada para calcular la funci\'on exponencial sumando los t\'erminos de esta serie hasta que "no haya cambio en la suma".
\item 

\begin{itemize}
\item Escribe c\'odigo en R que utiliza la funci\'on \texttt{seq ()} para generar un vector que contiene una secuencia num\'erica a partir de $0,05$ a $0,2$ en pasos de $0,05$ y asigna el resultado a un objeto llamado \texttt{pReg}.
\item Escribe c\'odigo en R para la siguiente expresi\'on matem\'atica:

\[
(1 -pReg)^{40}
\]
\item Anote en palabras lo que el resultado del siguiente c\'odigo en R, muestra (explica que tipo de estructura de datos es creada, que representa cada valor en la estructura)

\begin{knitrout}
\definecolor{shadecolor}{rgb}{0.969, 0.969, 0.969}\color{fgcolor}\begin{kframe}
\begin{alltt}
\hlstd{> }\hlstd{nJuegos} \hlkwb{<-}\hlkwd{seq}\hlstd{(}\hlnum{20}\hlstd{,} \hlnum{40}\hlstd{,} \hlnum{5}\hlstd{)}
\hlstd{> }\hlkwd{outer}\hlstd{(pReg, nJuegos,} \hlkwa{function}\hlstd{(}\hlkwc{p}\hlstd{,}\hlkwc{n}\hlstd{)\{}
\hlstd{+ }  \hlstd{(}\hlnum{1} \hlopt{-}\hlstd{p)}\hlopt{^}\hlstd{n}
\hlstd{+ }  \hlstd{\})}
\end{alltt}
\end{kframe}
\end{knitrout}
\end{itemize}
\item Escriba una funci\'on en R  llamada \texttt{nth} que, dado un vector $x$ de valores l\'ogicos y un entero positivo $n$, devuelva el \'indice del n-\'esimo valor verdadero  en $x$. Si hay menos de $n$ valores verdaderos en $x$, la funci\'on debe devolver NA. La función deber\'ia funcionar de la siguiente manera:

\begin{knitrout}
\definecolor{shadecolor}{rgb}{0.969, 0.969, 0.969}\color{fgcolor}\begin{kframe}
\begin{alltt}
\hlstd{> }\hlstd{x}\hlkwb{<-}\hlkwd{c}\hlstd{(}\hlnum{1}\hlstd{,}\hlnum{2}\hlstd{,}\hlnum{4}\hlstd{,}\hlnum{2}\hlstd{,}\hlnum{1}\hlstd{,}\hlnum{3}\hlstd{)}
\hlstd{> }\hlkwd{nth}\hlstd{(x} \hlopt{>}\hlnum{2}\hlstd{,} \hlnum{2}\hlstd{)}
\hlstd{> }\hlnum{6}
\hlstd{> }\hlkwd{nth}\hlstd{(x} \hlopt{>} \hlnum{4}\hlstd{,}\hlnum{2}\hlstd{)}
\hlstd{> }\hlnum{NA}
\end{alltt}
\end{kframe}
\end{knitrout}
\item Escribe funciones en R que llevan a cabo cada uno de los siguientes c\'alculos y proporcione \mbox{comentarios} apropiados para cada funci\'on.
\begin{itemize}
\item Escribe una funci\'on en R que, dada una matriz num\'erica $x$, devuelve un vector que contiene las desviaciones est\'andar de cada una de las columnas de $x$.
\item Escribe una función en R que, dada una matriz num\'erica $x$, devuelve un vector que contiene el promedio de los elementos  mayor y menor de cada fila de dicha matriz.
\item Escribe una funci\'on de R que calcula la media de una potencia dada de  elementos de un vector $x$. Adem\'as de $x$ la funci\'on  debe tener dos argumentos opcionales. El primero, \texttt{pow} que especifica la potencia y debe tener por defecto el valor 1 y el segundo, \texttt{na.rm}, que indica si o no los valores NA  deben ser omitidos cuando la media se calcula y que tiene por valor por defecto FALSE.
\item Un punto es un m\'aximo local de un vector si es mayor que sus vecinos inmediatos en el vector. (Los primeros y \'ultimos puntos pueden ser m\'aximos locales.) Escribe una funci\'on en  R que, dado un vector $x$ num\'erico (que no contiene valores), calcula los \'indices de los m\'aximos locales en $x$.
\end{itemize}
\item Usa las funciones \texttt{matrix()}, \texttt{seq()} y \texttt{rep()} para construir la matrices de Henkel $5 \times 5$.

\[
M = \begin{bmatrix}
        1 & 2 & 3  & 4 & 5         \\[0.3em]
        2 & 3 & 4 & 5 & 6           \\[0.3em]
        3 & 4 & 4 & 6 & 7           \\[0.3em]
       4 & 5 & 6 & 7 & 8           \\[0.3em]
        5 & 6 & 7 & 8 & 9           
     \end{bmatrix}
     \]
Convierte el c\'odigo en una funci\'on  que puede ser usado para construir matrices de dimensi\'on $n \times n$. Usa esa funci\'on para mostrar las salida de Matrices de Henkel de orden $10 \times 10$ y $12 \times 12$.

\item La matriz de Hilbert $n \times n$ tiene a los elementos $(i,j)$ dados por $1/(i + j-1)$.
\begin{itemize}
\item Escribe una funci\'on que muestra una matriz de Hilbert $n \times n$ como salida para entero positivo $n$.
\item ?` Son todas las matrices de Hilbert invertibles?.
\item Usa \texttt{solve()} y \texttt{qr.solve()} para calcular la inversa de las matrices Hilbert, por ejemplo, cuando $n = 10$.
\end{itemize}
\end{enumerate}
\end{document}
